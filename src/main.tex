\documentclass[10pt]{beamer}
\usetheme{metropolis}
% all imports
\input{all_imports}

\AtBeginEnvironment{quote}{\singlespacing}

% new commands
\input{all_new_commands}

% definitions
\input{definitions/colors}
\input{definitions/styles}

\title{Aprendizado de Máquina em Linguagem Natural}
\date{\today}

\author{
  Beatriz Albiero\\
  \url{https://github.com/beatrizalbiero}
  \vspace{0.1 cm}
  \and\\ 
  Felipe Salvatore\\
  \url{https://felipessalvatore.github.io/}
  \vspace{0.1 cm}
  \and\\
}

\institute{\textbf{USP}:University of São Paulo}


\begin{document}
\nocite{DeepLearningbook}
\maketitle

\section{Aprendizado de máquina}

\begin{frame}{Introdução Aprendizado de Máquina}

\begin{itemize}
\item ML é realizar uma tarefa baseado em dados.
\end{itemize}

Fazer desenho das coisas abaixo!!!!!!!!!!!!!!!

\begin{itemize}
\item \textbf{Regressão} $x_1, \dots, x_N \rightarrow y_1, \dots, y_N$.

\item \textbf{classificação} $x_1, \dots, x_N \rightarrow L_1, \dots, l_N$.
\end{itemize}
\end{frame}

\begin{frame}{Regressão}
colocar um plot falso mas verossivel !!!!!!!!!!!!!!!!!!!
\end{frame}


\begin{frame}{Classificação}
arrumar flechas !!!!!!!!!!!!!!!!!!!
\begin{figure}[ht!]
\centering

\scalebox{1.1}{
\begin{tikzpicture}[auto]

% operations =============================

% nodes
\node (shirt)
    {\includegraphics[width=.15\textwidth]{images/1fmnist.jpg}};
\node[textonly, right=30pt of shirt] (y1) {$y=$ camisa};

\node[below=10pt of shirt] (bag)
    {\includegraphics[width=.15\textwidth]{images/2fmnist.jpg}};
\node[textonly, right=30pt of bag] (y2) {$y=$ bolsa};

\node[right=10pt of y1] (dress)
    {\includegraphics[width=.15\textwidth]{images/3fmnist.jpg}};
\node[textonly, right=30pt of dress] (y3) {$y=$ vestido};

\node[below=10pt of dress] (sneaker)
    {\includegraphics[width=.15\textwidth]{images/4fmnist.jpg}};
\node[textonly, right=30pt of sneaker] (y4) {$y=$ tênis};






% edges
\path[tedge, orange!120, line width=1mm] (shirt) -- (y1);
\path[tedge, orange!120, line width=1mm] (bag) -- (y2);
\path[tedge, orange!120, line width=1mm] (dress) -- (y3);
\path[tedge, orange!120, line width=1mm] (sneaker) -- (y4);



\end{tikzpicture}
} % scalebox
\end{figure}

\end{frame}


\begin{frame}{Aprendizado por Redes Neurais}

O que sabemos sobre o cérebro:
\begin{itemize}
\visible<2->{\item neurônios em rede}
\visible<3->{\item  neurônios emitem sinais elétricos (disparam)}
\visible<4->{\item dendritos e axônios}

\end{itemize}
\end{frame}



\begin{frame}{Aprendizado por Redes Neurais}

Traduzindo para o modelo de redes neurais artificiais:
\begin{itemize}
\item Energia recebida: \alert{Input}
\item  Energia enviada: \alert{Output}
\item Carga mínima: \alert{Threshold}
\item Uma função que recebe um input e emite um output mas leva em consideração um threshold mínimo: \visible<2->{\alert{Função de ativação}}

\end{itemize}

\end{frame}



\begin{frame}{Função de ativação 1: sigmoid}
\input{tikzfiles/Sigmoid}
\end{frame}

\begin{frame}{função de ativação 2: relu}
\input{tikzfiles/ReLU}
\end{frame}

\begin{frame}[fragile]{perceptron}
\input{tikzfiles/perceptron}
\end{frame}

% As the name suggests, neural-networks are inspired by the brain’s computation mechanism, which consists of computation units called neurons. In the metaphor, a neuron is a computational unit that has scalar inputs and outputs. Each input has an associated weight.The neuron multiplies each input by its weight, and then sums them, applies a non-linear function to the result, and passes it to its output. While the brain metaphor is sexy and intriguing, it is also distracting and cumbersome to manipulate mathematically.


\begin{frame}[fragile]{Rede Neural}
\begin{figure}[ht!]
\centering

\scalebox{0.9}{
\begin{tikzpicture}[auto]

% operations =============================

% input layer
\node[op] (x3) {$x_3$};
\node[op, above=10.5pt of x3] (x2) {$x_2$};
\node[op, above=10.5pt of x2] (x1) {$x_1$};
\node[op, below=10.5pt of x3] (x4) {$x_4$};

% hidden layer
\node[state,  above right= 10pt and 90pt of x3] (h2) {$h_2$};
\node[state, below=10.5pt of h2] (h3) {$h_3$};
\node[state, above=10.5pt of h2] (h1) {$h_1$};

% output layer
\node[output,  right=190pt of x3] (out3) {$\hat{y}_3$};
\node[output,  right=190pt of x1] (out1) {$\hat{y}_1$};
\node[output,  right=190pt of x2] (out2) {$\hat{y}_2$};
\node[output,  right=190pt of x4] (out4) {$\hat{y}_4$};


% text only
\node[boxtextonly, above=6.5pt of x1] (entrada) {entrada\\ (\textit{input layer})};
\node[boxtextonly, above=6.5pt of h1] (escondida) {camada escondida\\ (\textit{hidden layer})};
\node[boxtextonly, above=6.5pt of out1] (saida) {sáida\\ (\textit{output layer})};

%edges input to hidden

\path[tedge, line width=1.2mm] (x1) -- (h1);
\path[tedge] (x1) -- (h2);
\path[tedge] (x1) -- (h3);

\path[tedge] (x2) -- (h1);
\path[tedge] (x2) -- (h2);
\path[tedge, line width=1.2mm] (x2) -- (h3);

\path[tedge] (x3) -- (h1);
\path[tedge, line width=1.2mm] (x3) -- (h2);
\path[tedge] (x3) -- (h3);

\path[tedge] (x4) -- (h1);
\path[tedge, line width=1.2mm] (x4) -- (h2);
\path[tedge] (x4) -- (h3);

% edges hidden to output
\path[tedge] (h1) -- (out1);
\path[tedge] (h1) -- (out2);
\path[tedge,line width=1.2mm] (h1) -- (out3);
\path[tedge] (h1) -- (out4);

\path[tedge] (h2) -- (out1);
\path[tedge,line width=1.2mm] (h2) -- (out2);
\path[tedge] (h2) -- (out3);
\path[tedge] (h2) -- (out4);

\path[tedge,line width=1.2mm] (h3) -- (out1);
\path[tedge] (h3) -- (out2);
\path[tedge] (h3) -- (out3);
\path[tedge,line width=1.2mm] (h3) -- (out4);





\end{tikzpicture}
} % scalebox
\end{figure}

\end{frame}

\begin{frame}[fragile]{Versão resumida de uma rede neural}
\input{tikzfiles/CompGraph}
\end{frame}

\begin{frame}{Resumo}
\begin{figure}[ht!]
\centering

\scalebox{0.8}{
\begin{tikzpicture}[auto]

% nodes =============================
\node[box] (ent) {entrada};
\node[box, right=40pt of ent] (NN) {modelo};
\node[box, right=40pt of NN] (saida) {saída};
\node[textonly, right=18pt of saida] (inv1) {};
\node[textonly, below=5pt of inv1] (inv2) {};
\node[box, right=40pt of saida] (saidaesp) {saída \\ esperada};
\node[boxtextonly, below=90pt of inv1] (comp) {Comparação \\ Erro};
\node[boxtextonly, left=70pt of comp] (Uso) {Alteração \\ dos \\ parâmetros};

% edges =============================
\path[tedge, orange!120, line width=1mm] (ent) -- (NN);
\path[tedge, orange!120, line width=1mm] (NN) -- (saida);
\path[tedge, orange!120, line width=0.5mm] (saida) -- (saidaesp);
\path[tedge, orange!120, line width=0.5mm] (saidaesp) -- (saida);
\path[tedge, orange!120, line width=1mm] (inv2) -- (comp);
\path[tedge, orange!120, line width=1mm] (comp) -- (Uso);
\path[tedge, orange!120, line width=1mm] (Uso) -- (NN);


% animation ============================

\visible<2->{\node[textonly, above=5pt of ent] (shirt) {\includegraphics[width=.10\textwidth]{images/shirt.png}};}

\visible<3->{\node[textonly, above=8pt of NN] (rede) {Rede Neural};}

\visible<4->{\node[textonly, above=8pt of saida] (saida2) {meia};}

\visible<5->{\node[textonly, above=8pt of saidaesp] (saida3) {camisa};}

\visible<6->{\node[textonly, below=15pt of comp] (ce) {Entropia cruzada};}

\visible<7->{\node[boxtextonly, left=15pt of Uso] (sgd) {Descida \\do \\Gradiente};}

\end{tikzpicture}
} % scalebox
\end{figure}

\end{frame}


\begin{frame}[fragile]{Descida do gradiente}
\begin{figure}
\centering
\includegraphics[width=1.0\linewidth]{images/Minimization_image.png}
\end{figure}
\end{frame}

\section{Aprendizado de máquina em linguística}

\begin{frame}[fragile]{On Learning the Past Tenses of English Verbs}

Problema:
\begin{itemize}
\item Aprendizado dos verbos "irregulares" no passado do inglês
\visible<2->{\item Famílias de verbos irregulares:}
	\visible<3->{\begin{itemize}
	\item “blow–blew, grow–grew, know–knew, throw–threw”
	\item “bind–bound, find–found, grind–ground, wind–wound”
	\item “drink–drank, shrink–shrank, sink–sank, stink–stank”}
	\end{itemize}
\visible<4->{\item Chomsky vs Rumelhart e McClelland}
\visible<5->{\item Aprendizado por Regras (Racionalismo) vs  Aprendizado por Analogias (Conexionismo)}
\end{itemize}

\end{frame}


\begin{frame}[fragile]{Rede Neural}
\begin{figure}[ht!]
\centering

\scalebox{0.63}{
\begin{tikzpicture}[auto]

% operations =========

% input
\node[normal] (x) {};

\node[textonly, left=5pt of x] (falar) {$\#$falar$\#$};


% phon features 1
\node[textonly, above right=20pt and 60pt of x] (1pho1) {int, vowel, int};

\node[textonly, below=20pt of 1pho1] (1pho2) {int, vowel, cont};


% FNN input
\node[normal, right=5pt of 1pho1] (x1) {};
\node[normal, below=5pt of x1] (x2) {};




% FNN output
\node[normal, right=45pt of x1] (y1) {};
\node[normal, right=45pt of x2] (y2) {};


% phon features 2
\node[textonly, right=5pt of y1] (2pho1) {int, vowel, int};

\node[textonly, right=5pt of y2] (2pho2) {int, vowel, cont};


% output
\node[normal, below right=20pt and 120pt of y1] (y) {};
\node[textonly, right=5pt of y] (falu) {$\#$falu$\#$};



% edges input -- FNN
\path[nedge_dashed] (x) -- (1pho1);
\path[nedge_dashed] (x) -- (1pho2);



% edges FNN
\path[nedge] (x1) -- (y1);
\path[nedge] (x1) -- (y2);
\path[nedge] (x2) -- (y1);
\path[nedge] (x2) -- (y2);


% edges output -- FNN
\path[nedge_dashed] (y) -- (2pho1);
\path[nedge_dashed] (y) -- (2pho2);




\end{tikzpicture}
} % scalebox
\end{figure}

\end{frame}


\begin{frame}[fragile]{On Learning the Past Tenses of English Verbs}

Exemplo de entrada ($x$) e saída ($y$):

\begin{itemize}
\item [] $(\vect{x}^{(1)}, \vect{y}^{(1)}) =$ (begin, began).
\item [] $(\vect{x}^{(2)}, \vect{y}^{(2)}) =$ (love, loved)
\item [] $(\vect{x}^{(3)}, \vect{y}^{(3)}) =$ (drink, drank)
\item [] $(\vect{x}^{(4)}, \vect{y}^{(4)}) =$ (hate, hated)
\item [] $(\vect{x}^{(5)}, \vect{y}^{(5)}) =$ (grow, grew)
\item [] $(\vect{x}^{(6)}, \vect{y}^{(6)}) =$ (bind, bound)
\item [] $(\vect{x}^{(7)}, \vect{y}^{(7)}) =$ (hit, hit)
\item [] $\dots$
\end{itemize}


\end{frame}


\begin{frame}[fragile]{On Learning the Past Tenses of English Verbs}

X será uma combinação de traços (features) fonológicos.

\begin{table}[]
\centering
\caption{Bia, lembre de colocar uma legenda}
\label{fontable}
\begin{tabular}{llllllll}
\hline
 &  & \multicolumn{6}{c}{Place} \\ \hline
 &  & \multicolumn{2}{c|}{Front} & \multicolumn{2}{c|}{Middle} & \multicolumn{2}{c}{Back} \\ \hline
 &  & \multicolumn{1}{l|}{V/L} & \multicolumn{1}{l|}{U/S} & \multicolumn{1}{l|}{V/L} & \multicolumn{1}{l|}{U/S} & \multicolumn{1}{l|}{V/L} & U/S \\ \hline
 \multicolumn{1}{c}{\multirow{2}{*}{Int.}} & Stop & b & p & d & t & g & k \\ \cline{2-8} 
\multicolumn{1}{c}{} & Nasal & m & - & n & - & N & - \\ \hline
\multirow{2}{*}{Cont} & Fric & v/D & f/T & z & s & Z/j & S/C \\ \cline{2-8} 
 & Liq/SV & w/l & - & r & - & y & h \\ \hline
 \multirow{2}{*}{Vowel} & High & E & i & O & - & U & u \\ \cline{2-8} 
 & Low & A & e & I & a/$\alpha$ & W & o \\ \hline
\end{tabular}
\end{table}
\end{frame}


\begin{frame}[fragile]{On Learning the Past Tenses of English Verbs}
\Large{
\begin{align*}
  \vect{x} &= \begin{bmatrix}
                         1\\
                         0\\
                         \vdots \\
                         1\\
                         \vdots \\
                         0
                         \end{bmatrix}\\
\end{align*}
 }
\end{frame}


\begin{frame}[fragile]{On Learning the Past Tenses of English Verbs}
explicar a saida do modelo e funcao de custo
\end{frame}


\begin{frame}[fragile]{On Learning the Past Tenses of English Verbs}
Resultados (PROS):

\begin{itemize}
\visible<2->{\item Identificou padrões corretamente entre todos os 420 verbos do treinamento;}
\visible<3->{\item Taxa de acerto de 92\% para verbos regulares ausentes no treinamento;}
\visible<4->{\item Taxa de acerto de 84\% para verbos irregulares ausentes no treinamento;}
\visible<5->{\item U-shaped Development}
\end{itemize}
\end{frame}

\begin{frame}[fragile]{On Learning the Past Tenses of English Verbs}
Resultados (CONS):

\begin{itemize}
\visible<2->{\item A rede falha ao tentar fazer predições com palavras que compartilham muitas features em comum; \\Exemplo: "algalgal" - Oykangand}
\visible<3->{\item Muitos problemas como uma teoria da mente}
\end{itemize}
\end{frame}



\begin{frame}[fragile]{As Irregularidades no Português Brasileiro}
Desafios:
\begin{itemize}
\visible<2->{\item "Wug Test"; \\ Exemplos: "poguir", "redir", "atover"}
\visible<3->{\item Adaptar a rede para a língua portuguesa}
\visible<4->{\item Melhorar o desempenho da rede}
\end{itemize}
\end{frame}

\section{Modelos de linguagem e redes recorrentes}


\begin{frame}{Definition}
We call \alert{language model} a probability distribution over sequences of tokens in a natural language.

\[
P(x_1,x_2,x_3,x_4) = p
\]

\textbf{Used for}:
\begin{itemize}
\item speech recognition
\item machine translation
\item text auto-completion
\item spell correction
\item question answering
\item summarization
\end{itemize}


\end{frame}

\begin{frame}{revisao probabilidade}

independencia, prop condicional

\end{frame}


\begin{frame}{How do we build these probabilities?}
Using the chain rule of probability: 

\begin{equation*}
P(x_1,x_2,x_3,x_4) = P(x_1)P(x_2\vert x_1)P(x_3\vert x_1x_2)P(x_4\vert x_1x_2x_3)
\end{equation*}

\vspace{0.3cm}

To make things simple we use a \textbf{Markovian assumption}, i.e., for a specific $n$ we assume that:

\begin{equation*}
P(x_1, \dots, x_T) = \prod_{t=1}^{T} P(x_t \vert x_1, \dots, x_{t-1}) = \prod_{t=1}^{T} P(x_{t} \vert x_{t - (n+1)}, \dots, x_{t-1})
\end{equation*} 

\end{frame}

\begin{frame}{Models based on $n$-gram statistics}
The choice of $n$ yields different models.\\

\textbf{Unigram} language model ($n=1$): 
\begin{equation*}
P_{uni}(x_1, x_2, x_3, x_4) = P(x_1)P(x_2)P(x_3)P(x_4)
\end{equation*}

where $P(x_i) = count(x_i)$.\\

\textbf{Bigram} language model ($n=2$): 
\begin{equation*}
P_{bi}(x_1,x_2,x_3,x_4) = P(x_1)P(x_2\vert x_1)P(x_3\vert x_2)P(x_4\vert x_3)
\end{equation*} 
where
\[
P(x_i\vert x_j) = \frac{count(x_i, x_j)}{count(x_j)}
\]
\end{frame}

\begin{frame}{$n$-gram statistics}
\url{https://books.google.com/ngrams}
\vspace{0.4cm}

\includegraphics[scale=0.14]{images/AI_ML.png}
\end{frame}



\begin{frame}{Models based on $n$-gram statistics}
\begin{itemize}
\item Higher $n$-grams yields better performance.
\vspace{0.7cm}
\item Higher $n$-grams requires a lot of memory!
\vspace{0.1cm}
\begin{quote}
"Using one machine \textbf{with 140 GB
RAM for 2.8 days}, we built an unpruned
model on 126 billion tokens."
\end{quote}
\begin{itemize}
\item [] \textit{Scalable Modified Kneser-Ney Language Model Estimation} by Heafield et al.
\end{itemize}
\end{itemize}
\end{frame}


\begin{frame}{Language model as sequential data prediction}
Instead of using one approach that is specific for the language domain, we can use a general model for sequential data prediction: a \textbf{RNN}. \\

Our learning task is to estimate the probability distribution 

\[
P(x_{n} = \text{word}_{j^{*}} | x_{1}, \dots ,x_{n-1})
\]

for any $(n-1)$-sequence of words $x_{1}, \dots ,x_{n-1}$.
\end{frame}


\begin{frame}[fragile]{RNNs}
\begin{figure}[ht!]
\centering

\scalebox{1.40}{
\begin{tikzpicture}[auto]

% RNN state cell =============================
\node[state] (h) {$\vect{h}$};
\node[op, below=30pt of h] (x) {$\vect{x}$};
\node[output, above=30pt of h] (yhat) {$\hat{\vect{y}}$};



% edges
\path[tedge] (x) edge node[below right= -4pt] {}  (h) ;
\path[tedge] (h) edge [out=-400,in=-320,looseness=8, distance=125pt] node[above right] {} (h);
\path[tedge] (h) edge node[below right = -4pt] {} (yhat);


\end{tikzpicture}
} % scalebox
\end{figure}

\end{frame}


\begin{frame}[fragile]{RNNs}
\input{tikzfiles/RNN2}
\end{frame}


\begin{frame}{Classificação com uma rede neural}
\input{tikzfiles/DFNclassification}
\end{frame}


\begin{frame}{Building the dataset}
We start with a corpus $C$ with $T$ tokens and a vocabulary $\Vocab$.\\\

Example: \textbf{Make Some Noise} by the Beastie Boys.\\

\begin{quote}
\alert{Yes, here we go again, give you more, nothing lesser\\
Back on the mic is the anti-depressor\\
Ad-Rock, the pressure, yes, we need this\\
The best is yet to come, and yes, believe this\\
... \\}
\end{quote}

\begin{itemize}
\item $T = 378$
\item $|\Vocab| = 186$
\end{itemize}

\end{frame}

\begin{frame}{Building the dataset}
The dataset is a collection of pairs $(\vect{x},\vect{y})$ where $\vect{x}$ is one word and $\vect{y}$ is the immediately next word. For example:
\begin{itemize}
\item [] $(\vect{x}^{(1)}, \vect{y}^{(1)}) =$ (Yes, here).
\item [] $(\vect{x}^{(2)}, \vect{y}^{(2)}) =$ (here, we)
\item [] $(\vect{x}^{(3)}, \vect{y}^{(3)}) =$ (we, go)
\item [] $(\vect{x}^{(4)}, \vect{y}^{(4)}) =$ (go, again)
\item [] $(\vect{x}^{(5)}, \vect{y}^{(5)}) =$ (again, give)
\item [] $(\vect{x}^{(6)}, \vect{y}^{(6)}) =$ (give, you)
\item [] $(\vect{x}^{(7)}, \vect{y}^{(7)}) =$ (you, more)
\item [] $\dots$
\end{itemize}
\end{frame}


\begin{frame}{The language model: graph}
\begin{figure}[ht!]
\centering

\scalebox{1.40}{
\begin{tikzpicture}[auto]

% RNN state cell =============================
\node[state] (h) {$\vect{h}$};
\node[op, below=30pt of h] (e) {$\vect{e}$};
\node[output, above=30pt of h] (yhat) {$\hat{\vect{y}}$};

\node[textonly, below right=0.5pt and 0.5pt of e] (inv1) {};
\node[textonly, above right=0.5pt and 0.5pt of e] (inv2) {};
\node[textonly, right=40pt of e] (embedding) {camada de embedding};





% edges
\path[tedge] (e) edge node[below right= -4pt] {}  (h) ;
\path[tedge] (h) edge [out=-400,in=-320,looseness=8, distance=125pt] node[above right] {$\vect{W}$} (h);
\path[tedge] (h) edge node[below right = -4pt] {} (yhat);


% % info edges
\draw[orange!120, line width=1mm]  (embedding) to [out=-180,in=0] (inv1);
\draw[orange!120, line width=1mm] (embedding) to [out=-180,in=0] (inv2);


\end{tikzpicture}
} % scalebox
\end{figure}

\end{frame}

\begin{frame}{The Language model: unfolding example}
\input{tikzfiles/LanguageModelUnfolded}
\end{frame}

\begin{frame}{Loss and Perplexity}
So another definition of perplexity is
\vspace{0.5cm}
\Large{
\begin{equation*}
2^{L} = PP(C)
\end{equation*}
}
\end{frame}


\begin{frame}{Vanishing}

If we initialize $\vect{W}$ such that $||\vect{W}|| < 1$, the gradient for further time steps will be very small (\alert{vanishing problem}).

\vspace{0.8cm}
 \url{https://www.youtube.com/watch?v=xAl8fu8myW0}


\end{frame}

\begin{frame}{Exploding}

If $||\vect{W}|| > 1$, the gradient for further time steps will be larger and larger (\alert{exploding problem}).

\vspace{0.8cm}
 \url{https://www.youtube.com/watch?v=dqW-jw5qKK8}


\end{frame}


\begin{frame}{The vanishing problem}

The gradients from the steps closed to $\tau$ (the last step) have more influence than the ones very far back.\\

This is bad for capturing \alert{long-term dependecies}.
\end{frame}

\begin{frame}{Possible solutions (hacks)}
\begin{itemize}
\item Clip gradients to a maximum value.
\vspace{0.4cm}
\item Choosing the right activation functions, e.g. ReLU.
\vspace{0.4cm}
\item Initialize weights to the identity matrix.
\vspace{0.4cm}
\item LSTM (Long Short-Term Memory), GRU (Gated Recurrent Unit), etc
\end{itemize}
\end{frame}

\begin{frame}{LSTM: recurrence}
\input{tikzfiles/LSTM_cell}
\end{frame}


\begin{frame}{MytwitterBot: TrumpBot}
\url{https://github.com/felipessalvatore/MyTwitterBot}
\begin{center}
\includegraphics[scale=0.24]{images/TrumpBot.png}
\end{center}
\end{frame}

\begin{frame}{MytwitterBot: SakaBot}
\url{https://github.com/felipessalvatore/MyTwitterBot}
\begin{center}
\includegraphics[scale=0.24]{images/SakaBot.png}
\end{center}
\end{frame}

\section{Conclusion}

\begin{frame}{What's next?}
After some experiments with the hyper parameters my best result on the \alert{Penn Treebank (PTB)} corpus was

\vspace{0.5cm}

\begin{figure}
\begin{center}
\begin{tabular}{|c|c|c|}
\hline
\cellcolor{blue!60}Model & \cellcolor{blue!60}Val & \cellcolor{blue!60}Test  \\ \hline
Mikolov et al (2011)\cite{Mikolov11} & $163.2$  & $149.9$ \\ \hline
\end{tabular}
\end{center}
\end{figure}
\end{frame}

\begin{frame}{Novas arquiteturas: \url{https://arxiv.org/abs/1708.02182}}

\begin{center}
\includegraphics[scale=0.34]{images/SocherPTB.png}
\end{center}

\end{frame}

\begin{frame}[allowframebreaks]{References}

  \bibliography{my_references}
  \bibliographystyle{abbrv}

\end{frame}

\end{document}